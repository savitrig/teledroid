\section{Motivation}
\label{sec:Motivation}
Modern mobile devices have significant synchronization requirements.  With storage in the gigabytes, these devices can carry a users' music collection, photos, videos, and other data.  As sensors and software progress, they're increasingly used to produce and download this material directly as well.  As this trend continues, the need for more frequent synchronization of this data between a user's mobile device and computer will become more pressing.

Furthermore, by keeping data synchronized between a low-performance mobile computer and a high performance desktop or server, new capabilities become available.  For example: the audio of a lecture recorded with the inexpensive microphones in a consumer mobile device will be alternately too quiet then too loud.  A single pass through a post-processor can level out the volume and make a tremendous difference, but it would take hours and surely drain the battery of the device.  By synchronizing with a powerful server it's not difficult to run an unmodified desktop application on the audio, and the results can then be automatically synchronized back to the device, all without the lengthy and obnoxious step of tethering to the desktop.

We performed a feasibility study of real-time, wireless synchronization.  By utilizing filesystem monitors present in the Android operating system and modern desktop operating systems we aimed to keep resource usage reasonable.