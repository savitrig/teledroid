\section{Performance Evaluation}
\label{sec:Performance}
We conducted our Teledroid application in three different mode: scan mode, monitor mode, and lazy mode. In scan mode, local device and server will output their own list of files along with their modified time. Then, Teledroid compares the two lists with the modified time to determine whether the file should be sync from one side to the other. In monitor mode, we enable inotify process on both server side and client side to monitor the changed file. Once the file is changed, inotify will report to Teledroid and request to sync that file. In lazy mode, even when the file has been changed on either side, that file will not be synced until it is read by Teledroid. In a limit of time, we can not complete the functionality for lazy mode. Thus, lazy mode will be our future work and was not be used in our experiment.
In our experiments, ......

Plan:\\
	- scan mode\\
	- monitor mode\\
	- lazy mode (Future work)\\
	
	sample:	
		- one Large-size file\\
		- multiple small-size files\\
		(with new or modified files)\\
		
Hardware Configuration\\

Android Dev Phone 1 was employed as our experiment tool. There are one Qualcomm 7210 processor in 528MHZ and 192 MB RAM memory in Android Dev Phone 1. With a touch screen and a trackball for navigation, it also provides QWERTY slider keyboard for input. Wi-Fi, GPS,  and Bluetooth v2.0 are all supported in Android Dev Phone 1. For network standard of cellular provider, it can support 3G WCDMA in 1700/2100 MHz and Quad-band GSM in 850/900/1800/1900 MHz. Note that Android Dev Phone 1 includes 1GB MicroSC card as an external hard drive. It can be replaced with up to 16GB card.
Concerning network environment, we proposed to connect to CSLabs network in University of San Francisco using Wi-Fi connection.

	- device: Android Dev Phone 1\\
	- network environment:\\
	- battery level\\
	- server setting\\

Graphs and statistics:\\ 
	- CPU\\
	- Memory\\
	- bandwidth usage\\

Explanation:  (Peter)\\